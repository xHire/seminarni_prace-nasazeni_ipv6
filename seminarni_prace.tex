\documentclass[12pt]{report}
\usepackage[utf8x]{inputenc}
\usepackage[myczech]{babel}
\usepackage{a4wide}
\linespread{1,5}		% 1,5 řádkování
\usepackage[hmargin={3cm,2cm},vmargin={2.5cm,2.5cm},includeheadfoot]{geometry}
\usepackage{indentfirst}	% odsazení prvního řádku v odstavci

% hezká záhlaví
\usepackage{fancyhdr}
\pagestyle{fancy}
\setlength{\headheight}{15pt}

% nastavení
\setcounter{chapter}{-1}	% aby úvod byl 0

% vlastní makra
\newcommand{\kapitola}[1]{
	\refstepcounter{chapter}
	\lhead{\thechapter. #1}
	\rhead{Michal Zima}
	\typeout{\chaptername\space\thechapter.}
	\chapter*{\protect\thechapter\hspace{0.75em} #1}
	\addcontentsline{toc}{chapter} {
		\protect\numberline{\thechapter}#1
	}
}
\newcommand\uv[1]{\quotedblbase #1\textquotedblleft}

% data do titulky
\title{Nasazení IPv6 v praxi}
\author{Michal Zima}
\date{}

\begin{document}
% titulka
\maketitle
\setcounter{page}{2}	% titulka se prý má také počítat...

% titulní list?

% prohlášení, místo a datum
\thispagestyle{empty}
Prohlašuji, že jsem svou seminární práci napsal samostatně a výhradně s použitím citovaných pramenů. Souhlasím se zapůjčováním práce a jejím zveřejňováním v souladu s licencí Creative Commons Attribution-Share Alike.

\bigskip	% vynechá řádek
\noindent V Třebíči dne \today \hspace{\fill}Michal Zima
\newpage{}

% zadání

% anotace/abstrakt - odborný souhrn nejdůležitějších myšlenek (dvojjazyčně)
\thispagestyle{plain}
\section*{Nasazení IPv6 v praxi\\Implementation of IPv6 in use}

\subsection*{Anotace}
Seminární práce se zabývá problematikou nasazení technologie IPv6 do reálného prostředí tak, aby se eliminovalo použití staré technologie IPv4 uvnitř malých sítí. Cílem je usnadnit pozdější přechod celosvětového Internetu na tuto novou technologii.

\subsection*{Anotation}
Abstract labour deals with problematic of implementation IPv6 technology into real environment so that use of old IPv4 technology would be eliminated. The goal is to make easier further transforming worldwide Internet to this new technology.

% seznam klíčových slov (dvojjazyčně)
\subsection*{Klíčová slova}
Internet, IPv4, IPv6, NAT, IP Maškaráda, TCP, UDP, ICMP, tunelování

\subsection*{Keywords}
Internet, IPv4, IPv6, NAT, IP Masquerading, TCP, UDP, ICMP, tunneling
\newpage{}

% obsah
\lhead{Obsah}
\rhead{Michal Zima}
\tableofcontents
\newpage

% vlastní text práce
\kapitola{Úvod}
Všechno na světě se vyvíjí a nejinak je tomu s počítačovými technologiemi -- konkrétně tedy se sítěmi a Internetem. Přenosové rychlosti úspěšně rostou, množství přenesených dat též. S tím souvisí i nárůst počtu připojených serverů a klientů do této sítě. Jenže servery potřebují \uv{být vidět}. Řada klientů zase chce \uv{být vidět}. Musí tedy mít veřejně routovatelnou IP adresu. Protokol IP ve verzi 4 má své mantinely ve formě 32 bitů, a přes ty se již nelze znovu přehoupnout (poprvé to bylo učiněno maškarádou, ale pouze za předpokladu, že klienti, kteří za ní budou schovaní, nebudou chtít být vidět), a proto na ně pomalu začínáme narážet. Řešením je pouze přechod na novou verzi IP protokolu, a to na verzi 6.

IPv6 je na světě již poměrně dlouho -- první koncepty se objevily již v devadesátých letech, ale přesto je jeho produkční nasazení zatím spíše ojedinělé, příp. se jedná o paralelní nasazení současně s IPv4. Přitom situace ve světě IPv4 není zrovna růžová -- nyní, na počátku roku 2009, nám zbývá již jen něco kolem půl miliardy volných IPv4 adres. A víc jich nebude -- budou již jen ubývat. Je možné předpokládat, že čím méně jich bude, tím více se budou společnosti snažit jich co nejvíce získat. Efekt davové paniky.

Je potřeba zajistit, aby lidé začali přecházet -- čím dříve, tím lépe. I kdyby nasazovali IPv6 pouze jako technologii doplňující stávající IPv4, byl by to pokrok. Ovšem stále by správci sítí museli udržovat a spravovat staré IPv4, takže by si práci neušetřili, ba naopak by jim práce přibylo zhruba dvojnásobně. U velkých sítích tedy nelze hledat řešení v této cestě. Je potřeba nasadit čisté IPv6 se vším, co přináší, a využít tak všech jeho výhod naplno. Je samozřejmé, že je potřeba ještě nějaký čas zajišťovat klientům přístup do IPv4 Internetu, takže celý přechod nebude zas až tak jednoduchý. Ovšem jde to, a to je to hlavní, a proto nám již nic nebrání IPv6 plně nasadit.

IPv6 je velkou výzvou, a proto jsem se rozhodl tento nový protokol v domácí síti nasadit. A to bez sekundance staré IPv4. Ačkoliv by se to na první pohled mohlo jevit jako velmi snadný úkol, nebylo to tak docela triviální. Veškeré své postřehy, nabyté zkušenosti i řešení nastalých problémů shrnuji v této seminární práci.
\newpage{}

\kapitola{Výhody a nevýhody IPv6}
Ačkoliv je IPv6 nástupcem za starou verzi 4, nepřináší pouze výhody, ale oproti \uv{čtyřce} má i pár nevýhod. A na některé vlastnosti tohoto protokolu lze pro změnu nahlížet jako na výhodu i nevýhodu.

\section{Výhody}
\subsection{Velké množství adres}
Mezi hlavní výhody této verze \textit{Internetového Protokolu} patří především nesrovnatelně vyšší adresní rozsah, který lze použít. Namísto $2^{32}$ IPv4 adres najednou získáváme $2^{128}$ IPv6 adres. Tolik proklínaná maškaráda (běžněji známá jako tzv. NAT) se může velmi jednoduše stát minulostí.

\subsection{Autokonfigurace klientů}
IPv6 má velmi dobře vyřešenou schopnost autokonfigurace koncových klientů, takže v~mnoha případech odpadá (na první pohled) složitá konfigurace a vše funguje prakticky okamžitě -- stačí nastavit router.

\subsection{Menší nároky na výkon}
IPv6 protokol klade menší nároky na výpočetní výkon routerů, a to hned z několika důvodů: jednak není potřeba provádět maškarádu (překlad adres), a za další se pakety po cestě nefragmentují, takže routery nemusí pokaždé přepočítávat kontrolní součty (paket buď projde nebo bude zahozen -- samozřejmě s odesláním příslušného informačního ICMP paketu odesílateli).

\subsection{Zjednodušení hlavičky}
Hlavička IPv6 protokolu prošla přísnou revizí a mnoho zbytečných polí bylo odstraněno nebo nahrazeno -- z původních 12 zbylo pouze 8. Je to dáno také tím, že se o něco méně počítá s fragmentací paketů (neboť ji provádí pouze odesílatel), takže ta je přesunuta do rozšiřujících hlaviček.

\section{Nevýhody}
\subsection{Horší zapamatovatelnost adres}
Již od pohledu je na IPv6 vidět, že má velmi dlouhé adresy, a tím pádem i mnohem hůře zapamatovatelné než mělo IPv4. Je to daň za 128 bitů, ale vzhledem k možnosti zkracování (vynechávání počátečních nul v jednotlivých blocích, příp. vypouštění nulových bloků) je adresa často daleko jednodušší.

\subsection{Obtížnější boj se SPAMmery}
Mnohem více adres globálně znamená i mnohem více adres na osobu. Tedy i SPAMmeři jich dostanou obrovské množství. Bohužel se tím značně ztíží efektivita jejich blokování podle IP adres -- jedinou možností, jak se s tím vypořádat, je začít blokovat celé subnety i s tím rizikem, že tím mohou být postiženi i slušní uživatelé.

\subsection{Nutnost nových utilit}
Řada základních programů pro práci se sítí není s IPv6 přímo kompatibilní, a proto vznikly \uv{šestkové} verze těchto programů, přičemž jim byla přidána šestka na konec názvu, např. \textit{ping6}, \textit{traceroute6}. Někdy stará verze programu zastará a je jednoduše nahrazena novější, která IPv6 umí. V případě programu \textit{ifconfig} došlo přímo k nahrazení jiným programem (\textit{ip}) z balíku \textit{iproute2}.

\subsection{Větší IP hlavička}
IPv6 protokol má trošku větší hlavičku protokolu, a to o 20 B, což není moc, když vezmeme v úvahu, že potřebný datový prostor vzrostl o 24 B. Je tedy vidět, že formát hlavičky má za sebou pročištění a značné zjednodušení. Teoreticky by mohl v důsledku větší hlavičky trochu narůst globální traffic, ale ten poroste spíše z jiných důvodů.
\newpage{}

\kapitola{Důvody nasazení IPv6}
\section{Konec IPv4 na dohled}
Podle odhadů společnosti Intec NetCore, Inc. dojde adresní rozsah IPv4 na začátku roku 2011. Svět má tedy přibližně dva roky, aby se plně připravil na IPv6. V případě ignorování této \uv{hrozby} je možné očekávat vážné problémy.

\subsection{IPv4 adres je málo}
Nedostatek IPv4 adres je obecně považován za hnací motor a motivaci k přechodu na novou verzi IP protokolu. Vše ostatní je pak pouze považováno za příjemný vedlejší efekt. Dohromady ovšem dávají dobrý důvod, proč jít od IPv4 \uv{pryč}.

\subsection{IPv4 je drahý \uv{luxus}}
Dnes se prakticky za každou přidělenou IPv4 adresu platí velké peníze. Nejvíce to pociťují klienti ISP, webmasteři webů, které potřebují mít SSL zabezpečení, ale i samotní provozovatelé serverů. Jinak řečeno se dnes IPv4 adresy vyvažují takřka zlatem. A to proto, aby se zabránilo davové panice \uv{syslení} ještě volných adres, což by vedlo k jejich rychlému vyčerpání.

\section{IPv6 je dobré řešení}
\subsection{Rozdíl v ceně}
Zmínil jsem vysokou cenu za každou jednotlivou IPv4 adresu. Ve světě IPv6 naopak každý rozdává celé prefixy adres -- většinou druhou mocninu velikosti celého IPv4! 
Oproti IPv4 je IPv6 vskutku levné: jedinou investicí je implementace IPv6 do sítě, příp. obměna zastaralých síťových zařízení, které tento nový protokol nepodporují. Dál už si mohou všichni užívat přehršele volných adres.

\subsection{Malé množství přechodů}
Je až s podivem, že se přechody na novou verzi málokdo zabývá. Přitom motivace ve~formě tolika výhod je obrovská. Pravděpodobně se většina lidí nového protokolu bojí a snaží se jeho nasazení odkládat, jak jen to půjde, leč je takové řešení velmi krátkozraké a může se nevyplatit. Je ovšem též možné, že jen nevědí, jak přejít. Nechtějí řešit IPv4 i IPv6 v~jedné síti a jiné řešení ještě není všeobecně známo (čistá IPv6 síť by pro ně v tento moment znamenala odříznutí od velké části světového Internetu, což je neakceptovatelná podmínka), přičemž oni ho vymýšlet nechtějí.
\newpage{}

\kapitola{Předpoklady nasazení IPv6}
\section{Operační systém}
Základním předpokladem pro úspěšné nasazení IPv6 je vlastnictví IPv6 kompatibilního operačního systému. To je základ, bez kterého se za žádných okolností neobejdeme. Příslušná IPv6 konektivita je vždy až na druhém místě, stejně tak podpora IPv6 v aplikacích či síťových prvcích.

\subsection{GNU/Linux}
GNU/Linux byl vůbec první operační systém na světě, který IPv6 implementoval, a to již v roce 1996 ve verzi 2.1.8. Status experimentální implementace byl odstraněn v roce 2005 ve verzi 2.6.12. Od té doby je IPv6 plně podporované a funkční.

\subsection{Apple Mac OS X a jiné BSD systémy}
Do BSD systémů byla již v roce 2000 zařazena kvalitní implementace IPv6 projektu KAME, který byl započat roku 1998. Následně od roku 2003 je v Apple Mac OS X ve~výchozím nastavení IPv6 zapnuté.

\subsection{Microsoft Windows}
Ačkoliv začal Microsoft na podpoře IPv6 pracovat již v roce 1998, implementaci vhodnou do produkčního nasazení uvedl až ve verzi Vista v roce 2007. Napůl funkční implementace se též nachází v XP SP1 až SP3 z roku 2002, ovšem v praxi jí nelze využít, pokud klient současně nemá k dispozici i staré IPv4.

\section{Aplikace}
V případě, že klient má kompatibilní operační systém, zbývá ověřit podporu IPv6 v~samotných aplikacích. Většina moderních programů, které ke svému provozu potřebují připojení k Internetu, či které jsou na něm přímo postavené, dnes IPv6 podporují nebo mají alespoň své alternativy. Řádově slabší je ovšem situace okolo síťových her. Málokterá hra umí své datové toky řídit i na IPv6. Natož pak umožnit vytvoření herního serveru na IPv6, na který by se mohli hráči s IPv6 připojovat. Pro náruživé hráče může tento fakt znamenat jasné rozhodnutí mezi starým IPv4 a novým IPv6.

\section{Síťové prvky}
Možnost nativního nasazení IPv6 (tedy bez nutnosti tunelovat každého jednotlivého klienta) velmi často stojí a padá na jeho podpoře v aktivních síťových prvcích -- především pak routerech. Zvláště zařízení staršího data, která dlouho řadu let spolehlivě slouží svému původnímu účelu, se neumí vypořádat s IPv6, což činí síť z hlediska IPv6 nepoužitelnou.

U zařízení některých výrobců je možnost nahrát nový systém (\textit{firmware}), který již má podporu IPv6 implementovanou. U starších zařízení je takovýto firmware velmi často neoficiální a výrobcem nepodporovaný (a především nedoporučovaný). Přesto je to mnohdy jediné možné řešení, pomineme-li nákup nového zařízení.
\newpage{}

\kapitola{Proces nasazení IPv6}
\section{Přivedení IPv6 konektivity}
Provozovat čistou IPv6 lze i bez internetové IPv6 konektivity, ale v takovém případě si člověk říká o problémy. Řada aplikací totiž při existenci IPv6 v síti předpokládá funkčnost této sítě a preferuje ji před IPv4.

V praxi jsou nejčastější dva způsoby přivedení IPv6 konektivity, přičemž jejich rozšíření se liší od segmentu nasazení. Jednou z možností je požádat stávajícího ISP o přidělení IPv6 rozsahu. Tato možnost se u běžných ISP příliš nevyskytuje, takže lidé šahají po druhé možnosti. Tou je registrace IPv6 vlastního tunelu. Tunelování se naopak v praxi téměř nevyužívá v~serverovém segmentu, protože většina datacenter již vlastní nativní IPv6 konektivitou disponuje. Z řečeného vyplývá, že běžný uživatel bude přivedení IPv6 konektivity řešit tunelem, a proto se této problematice budu dále věnovat.

\subsection{Zřízení IPv6 tunelu}
Existuje velké množství firem, které se rozhodly zdarma poskytovat plnohodnotné IPv6 tunely (jsou to tzv. \uv{tunnel brokeři}). V základu se takto dá \uv{pořídit} konektivita např. do Německa (není mi známo, že by v České republice existoval veřejně přístupný tunnel broker) s velice slušnými odezvami (cca o 30 ms vyšší než normálně) a prefixem /64, příp. i /48. Pro malé nasazení považuji prefix /48 za zbytečný, protože pro automatickou bezstavovou konfiguraci /64 stačí. Při manuální konfiguraci či při použití DHCPv6 je i takto velký prefix naprosto nevyužitelný a v praxi by stačil i prefix o velikosti /120.

Tunnel brokerů je velké množství, přičemž já se budu zabývat pouze jedním, a to společností \textit{Hurricane Electric}, která provozuje \textit{http://www.tunnelbroker.net/}. Registrace je velmi rychlá a jednoduchá.

\begin{enumerate}
\item Nejprve vyplníme registrační formulář na adrese \textit{http://tunnelbroker.net/register.php}, a to pravdivými údaji, a především platnou e-mailovou adresou.
\item Na zadaný e-mail nám obratem přijde potvrzení registrace i s potřebným loginem a heslem.
\item Přihlásíme se.
\item Zažádáme o vytvoření tunelu na adrese \textit{http://tunnelbroker.net/ipv6\_normal.php} pod odkazem \uv{Create Regular Tunnel}. Vyplníme zde naši veřejnou IPv4 adresu a vybereme vhodný server pro druhý konec našeho tunelu. Doporučuje se vybrat ten, co je geograficky nejblíže, ale nemusí to platit vždy (důležité jsou odezvy -- čím menší, tím lepší), ovšem pro Českou republiku to bude vždy nějaký evropský server.
\item Hotovo. Náš tunel je připravený k použití.
\end{enumerate}

\subsection{Nastavení klientského počítače}
Na stránce s detaily tunelu se úplně dole skrývá \uv{Example OS Configurations} -- stačí tedy jen vybrat požadovaný operační systém a získáme přesné a hlavně personalizované nastavení, přesně krok za krokem. Nutno však podotknout, že se toto nastavení po restartu vrátí zpět (tedy se zruší), takže je velmi vhodné napsat si skript, který se při startu systému vždy spustí a toto nastavení obnoví.

\subsection{Otestování funkčnosti}
Nejlepší na otestování funkčnosti či nefunkčnosti tunelu, je program \textit{ping}, resp. jeho \uv{šestková} verze, \textit{ping6}. Jako první je dobré zkusit adresu \uv{Server IPv6 address} uvedenou na webu a následně adresu známého serveru, např. \textit{nic.cz}, \textit{ipv6.google.com} atd.

\section{Problémy s čistě IPv6 adresací}
Nasazení čistého IPv6 do dnešního internetového prostředí má několik málo překážek. Hned první je potřeba IPv6 DNS cache serveru, který bude zajišťovat v novém prostředí správný překlad doménových jmen na IP adresy. Druhý, ale řádově závažnější, problém je s nepřístupností stávajícího IPv4 Internetu v takové síti. IPv6 není zpětně kompatibilní s~IPv4 v takovém měřítku, aby bylo možné přistupovat z IPv6 do IPv4. Tento problém byl až dosud zcela zásadní otázkou u všech firem i ISP, kteří o \uv{výměně} za novou verzi protokolu uvažovali.

\section{IPv6 kompatibilní DNS cache}
\subsection{Klasická DNS cache}
Pokud člověk do sítě implementuje \uv{obyčejné} IPv6, může použít prakticky jakýkoliv IPv6 kompatibilní DNS server s podporou cachování. Jedním z možných kandidátů je například \textit{BIND}. Vzhledem k tomu, že veškeré technologické novinky jsou většinou implementovány v~BINDu jako první, tak i IPv6 plně podporuje.

\subsection{DNS cache s podporou speciální translace}
Já jsem ovšem šel dále než běžná řešení a naprogramoval jsem si vlastní DNS server, který jednak zprostředkovává překlad doménových jmen, ale především slouží jako prostředek ke~zprovoznění translace IPv6 do IPv4, kterou se zabývám v následující podkapitole. Tento server je součástí mého projektu \textit{WrapSix}.

\subsubsection{Základní princip}
Pokud se klient DNS serveru dotáže na obyčejný A záznam (tedy IPv4 adresu doménového jména), chová se tento server zcela podle očekávání. Pokud se dotáže na AAAA záznam (tedy IPv6 adresu doménového jména), a tento existuje, tak ho podle očekávání vrátí. Pokud ovšem nastane situace, kdy se klient dotáže na AAAA záznam, který neexistuje, tak se server přeptá i na A záznam -- pokud existuje, tak ho přeloží do potřebného formátu a vloží na konec nastaveného IPv6 prefixu.

\subsubsection{Výsledný efekt}
Ve výsledku tedy klient vždy (pokud je daný server dostupný) na požadavek o AAAA záznam tento záznam obdrží a bude tedy veškerý svůj provoz směrovat po novém IPv6. Díky tomu je možné IPv4 uvnitř sítě zcela vypustit.

\section{Překlad IPv6 na IPv4 a zpět}
Aby bylo výše uvedené řešení plně funkční, je potřeba dodat ještě druhý článek, druhou část WrapSixu. Tato druhá část se chová jako překladač IPv6 provozu, který přijde na~adresu z nastaveného IPv6 prefixu, do starého IPv4 (z toho vyplývá, že nutně potřebuje jeho existenci~-- např. na samotné hranici sítě). Případný vracející se IPv4 provoz správně vrací příslušným klientům uvnitř IPv6.

Díky tomuto řešení je opravdu možné IPv4 uvnitř sítě vypnout. Zajišťuje naprostou připravenost sítě na příchod čistého IPv6 Internetu a globálního vypnutí IPv4. Odfiltruje totiž IPv6 nekompatibilní operační systémy, aplikace i nevyhovující síťové prvky. Firmy, ISP i všichni lidé díky tomu mají možnost se přizpůsobit budoucím podmínkám postupně, rozložit investice raději do delšího časového horizontu, než poté chtít vše stihnout na poslední chvíli, kdy již bude éra IPv4 na pokraji své existence.

\section{Nepokryté oblasti}
Přesto toto řešení ještě není kompletní. Řada aplikací, ale především her, se připojuje přímo na IPv4 adresy serverů -- vynechává tedy mezikrok překladu doménového jména. Zatím tato oblast zůstává otevřená, ale brzy i ji WrapSix zaplní a vytvoří tak kompletní řešení pro každého (vzhledem k jeho svobodné open-source licenci).
\newpage{}

\end{document}
