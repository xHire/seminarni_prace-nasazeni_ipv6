\documentclass[12pt]{report}
\usepackage[utf8x]{inputenc}
\usepackage[czech]{babel}
\usepackage{a4wide}
\linespread{1,5}		% 1,5 řádkování
\usepackage[hmargin={3cm,2cm},vmargin={2.5cm,2.5cm},includeheadfoot]{geometry}
\usepackage{indentfirst}	% odsazení prvního řádku v odstavci

% hezká záhlaví
\usepackage{fancyhdr}
\pagestyle{fancy}

% nastavení
\setcounter{chapter}{-1}	% aby úvod byl 0

% vlastní makro
\newcommand{\kapitola}[1]{
	\refstepcounter{chapter}
	\lhead{\thechapter. #1}
	\rhead{Michal Zima}
	\typeout{\chaptername\space\thechapter.}
	\chapter*{\protect\thechapter\hspace{0.75em} #1}
	\addcontentsline{toc}{chapter} {
		\protect\numberline{\thechapter}#1
	}
}

% data do titulky
\title{Nasazení IPv6 v praxi}
\author{Michal Zima}
\date{}

\begin{document}
% titulka
\maketitle
\setcounter{page}{2}	% titulka se prý má také počítat...

% titulní list?

% prohlášení, místo a datum
\thispagestyle{empty}
Prohlašuji, že jsem svou seminární práci napsal samostatně a výhradně s použitím citovaných pramenů. Souhlasím se zapůjčováním práce a jejím zveřejňováním v souladu s licencí Creative Commons Attribution-Share Alike.

\bigskip	% vynechá řádek
\noindent V Třebíči dne \today \hspace{\fill}Michal Zima
\newpage{}

% zadání

% anotace/abstrakt - odborný souhrn nejdůležitějších myšlenek (dvojjazyčně)
\thispagestyle{plain}
\section*{Nasazení IPv6 v praxi\\Implementation of IPv6 in use}
\subsection*{Anotace}
Seminární práce se zabývá problematikou nasazení technologie IPv6 do reálného prostředí tak, aby se eliminovalo použití staré technologie IPv4 uvnitř malých sítí. Cílem je usnadnit pozdější přechod celosvětového Internetu na tuto novou technologii.
\subsection*{Anotation}
Abstract labour deals with problematic of implementation IPv6 technology into real environment so that use of old IPv4 technology would be eliminated. The goal is to make easier further transforming worldwide Internet to this new technology.
% seznam klíčových slov (dvojjazyčně)
\subsection*{Klíčová slova}
Internet, IPv4, IPv6
\subsection*{Keywords}
Internet, IPv4, IPv6
\newpage{}

% úvod

% obsah
\lhead{Obsah}
\rhead{Michal Zima}
\tableofcontents
\newpage

% vlastní text práce
\kapitola{Úvod}
Než se ponořím do nového světa IPv6, tak bych se pár slovy zmínil o mé motivaci, která mě vedla k napsání této práce. IPv6 už je na světě velmi dlouho (první koncepty se objevily již v devadesátých letech), ale přesto je jeho produkční nasazení spíše sporadické nebo jen jako vedlejší. Přitom situace na poli IPv4 není růžová - nyní, na počátku roku 2009, nám zbývá již jen něco kolem půl miliardy volných IPv4 adres.
\newpage{}

\kapitola{Výhody a nevýhody IPv6}
Jako každá technologie, tak i IPv6 má své výhody i nevýhody, které nelze nezmínit.
\section{Výhody}
Mezi hlavní výhody této verze Internetového Protokolu patří především nesrovnatelně vyšší adresní rozsah, který lze použít. Namísto $2^{32}$ IPv4 adres najednou získáváme $2^{128}$ IPv6 adres. Tolik proklínaná maškaráda (běžněji známá jako tzv. NAT) se může velmi jednoduše stát minulostí.

IPv6 protokol klade menší nároky na výpočetní výkon routerů, a to hned z několika důvodů: jednak není potřeba provádět maškarádu (překlad adres), a za další se pakety po cestě nefragmentují, takže routery nemusí pokaždé přepočítávat kontrolní součty (paket buď projde nebo bude zahozen - samozřejmě s odesláním příslušného informačního ICMP paketu odesílateli).
\section{Nevýhody}
IPv6 protokol má trošku větší hlavičku protokolu, a to o 20 B, což není moc, když vezmeme v úvahu, že potřebný datový prostor vzrostl o 24 B. Je tedy vidět, že formát hlavičky má za sebou pročištění a značné zjednodušení.
\newpage{}

\kapitola{Důvody nasazení IPv6}
Podle odhadů společnosti Intec NetCore, Inc. dojde adresní rozsah IPv4 na začátku roku 2011. Svět má tedy přibližně dva roky, aby se plně připravil na IPv6. V případě ignorování této "hrozby" je možné očekávat vážné problémy.
\newpage{}

\kapitola{Předpoklady nasazení IPv6}
Základním předpokladem pro nasazení IPv6 je IPv6 kompatibilní operační systém. GNU/Linuxové systémy tuto podmínku bez výhrad splňují. Operační systém Microsoft Windows musí být nejméně ve verzi Vista, protože žádná z předchozích verzí nemá použitelnou implementaci nové verze IP.
\newpage{}

\kapitola{Proces nasazení IPv6}
\section{Přivedení IPv6 konektivity}
V praxi jsou nejčastější dva způsoby přivedené IPv6 konektivity. Jednou možností je požádat stávajícího ISP o přidělení IPv6 rozsahu. Druhou pak zaregistrování IPv6 tunelu. Jelikož je získání IPv6 rozsahu od ISP mnohdy velmi problematické (např. z toho důvodu, že sám IPv6 nemá implementováno) a silně individuální, budu se dále věnovat jen "tunelování".
\subsection{Zřízení IPv6 tunelu}
\section{Problémy se samostatnou IPv6 adresací}
\section{IPv6 kompatibilní DNS cache}
\section{Převaděč IPv6 na IPv4 a naopak}
\section{Nepokryté oblasti}
\newpage{}

\end{document}
